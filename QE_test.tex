
\documentclass[11pt]{article}
\usepackage[utf8]{inputenc}
\usepackage{amsmath,amssymb,amsthm}
\usepackage{mathtools}
\usepackage{microtype}
\usepackage{enumitem}
\usepackage{geometry}
\geometry{margin=1in}
\title{Quantifier Elimination in Dense Algebraic/Ordered Structures}
\author{(Converted from PDF)}
\date{}


\newcommand{\BB}{\mathbb{B}}
\newcommand{\CC}{\mathbb{C}}
\newcommand{\DD}{\mathbb{D}}
\newcommand{\EE}{\mathbb{E}}
\newcommand{\FF}{\mathbb{F}}
\newcommand{\GG}{\mathbb{G}}
\newcommand{\HH}{\mathbb{H}}
\newcommand{\II}{\mathbb{I}}
\newcommand{\JJ}{\mathbb{J}}
\newcommand{\KK}{\mathbb{K}}
\newcommand{\LL}{\mathbb{L}}
\newcommand{\MM}{\mathbb{M}}
\newcommand{\NN}{\mathbb{N}}
\newcommand{\OO}{\mathbb{O}}
\newcommand{\PP}{\mathbb{P}}
\newcommand{\QQ}{\mathbb{Q}}
\newcommand{\RR}{\mathbb{R}}
%\newcommand{\SS}{\mathbb{S}}
\newcommand{\TT}{\mathbb{T}}
\newcommand{\UU}{\mathbb{U}}
\newcommand{\VV}{\mathbb{V}}
\newcommand{\WW}{\mathbb{W}}
\newcommand{\XX}{\mathbb{X}}
\newcommand{\YY}{\mathbb{Y}}
\newcommand{\ZZ}{\mathbb{Z}}


\newcommand{\cA}{\mathcal{A}}
\newcommand{\cB}{\mathcal{B}}
\newcommand{\cC}{\mathcal{C}}
\newcommand{\cD}{\mathcal{D}}
\newcommand{\cE}{\mathcal{E}}
\newcommand{\cF}{\mathcal{F}}
\newcommand{\cG}{\mathcal{G}}
\newcommand{\cH}{\mathcal{H}}
\newcommand{\cI}{\mathcal{I}}
\newcommand{\cJ}{\mathcal{J}}
\newcommand{\cK}{\mathcal{K}}
\newcommand{\cL}{\mathcal{L}}
\newcommand{\cM}{\mathcal{M}}
\newcommand{\cN}{\mathcal{N}}
\newcommand{\cO}{\mathcal{O}}
\newcommand{\cP}{\mathcal{P}}
\newcommand{\cQ}{\mathcal{Q}}
\newcommand{\cR}{\mathcal{R}}
\newcommand{\cS}{\mathcal{S}}
\newcommand{\cT}{\mathcal{T}}
\newcommand{\cU}{\mathcal{U}}
\newcommand{\cV}{\mathcal{V}}
\newcommand{\cW}{\mathcal{W}}
\newcommand{\cX}{\mathcal{X}}
\newcommand{\cY}{\mathcal{Y}}
\newcommand{\cZ}{\mathcal{Z}}
%
\newcommand{\bA}{\mathbb{A}}
\newcommand{\bB}{\mathbb{B}}
\newcommand{\bC}{\mathbb{C}}
\newcommand{\bD}{\mathbb{D}}
\newcommand{\bE}{\mathbb{E}}
\newcommand{\bF}{\mathbb{F}}
\newcommand{\bG}{\mathbb{G}}
\newcommand{\bH}{\mathbb{H}}
\newcommand{\bI}{\mathbb{I}}
\newcommand{\bJ}{\mathbb{J}}
\newcommand{\bK}{\mathbb{K}}
\newcommand{\bL}{\mathbb{L}}
\newcommand{\bM}{\mathbb{M}}
\newcommand{\bN}{\mathbb{N}}
\newcommand{\bO}{\mathbb{O}}
\newcommand{\bP}{\mathbb{P}}
\newcommand{\bQ}{\mathbb{Q}}
\newcommand{\bR}{\mathbb{R}}
\newcommand{\bS}{\mathbb{S}}
\newcommand{\bT}{\mathbb{T}}
\newcommand{\bU}{\mathbb{U}}
\newcommand{\bV}{\mathbb{V}}
\newcommand{\bW}{\mathbb{W}}
\newcommand{\bX}{\mathbb{X}}
\newcommand{\bY}{\mathbb{Y}}
\newcommand{\bZ}{\mathbb{Z}}



% sub- and superscripts
\newcommand{\an}{ \mathrm{an} }
\newcommand{\val}{ \mathrm{val} }
\newcommand{\ring}{ \mathrm{ring} }
\newcommand{\og}{ \mathrm{og} }
\newcommand{\Pres}{ \mathrm{Pres} }
\newcommand{\gDP}{\mathrm{gDP}}
\newcommand{\DP}{ \mathrm{DP}}
\newcommand{\Hen}{\mathrm{Hen}}
\newcommand{\alg}{\mathrm{alg}}
\newcommand{\trans}{\mathrm{trans}}
\newcommand{\fine}{\mathrm{fine}}
\newcommand{\oag}{\mathrm{oag}}
\newcommand{\ovf}{\mathrm{ovf}}
\newcommand{\trop}{\mathrm{trop}}
\newcommand{\RRV}{\mathrm{RRV}}
\newcommand{\omin}{\mathrm{omin}}
\newcommand{\eq}{\mathrm{eq}}
%

% Sorts
\newcommand{\RV}{\mathrm{RV}}
\newcommand{\RF}{\mathrm{RF}}
\newcommand{\RVprod}{ \RV_{\bar{n}} }
\newcommand{\VG}{\mathrm{VG}}
\newcommand{\VF}{\mathrm{VF}}
%

% Languages
\newcommand{\lang}{\mathcal{L}}
\newcommand{\Lval}{\lang_{\val}}
\newcommand{\Lor}{\lang_{\mathrm{or}}}
\newcommand{\Lan}{\lang_{\an}}
\newcommand{\Lring}{\lang_{\ring}}
\newcommand{\Lvf}{\lang_{\VF}}
\newcommand{\LRV}{ \lang^{\RV} }
\newcommand{\LRVplus}{\lang^{\RV^{+}}}
\newcommand{\Lac}{\lang^{\ac}}
\newcommand{\Log}{\lang_{\og}}
\newcommand{\Lpres}{\lang_{\Pres}}
\newcommand{\LDP}{\lang_{\DP}}
\newcommand{\LgDP}{\lang_{\gDP}}
\newcommand{\Lgroup}{\lang_{\mathrm{group}} }
\newcommand{\Ldpr}{\lang_{\gDP}^{\mathrm{res}}}
\newcommand{\Lrfn}{\lang_{\RF_N}}
\newcommand{\Lvg}{\lang_{\VG}}
\newcommand{\Lrv}{\lang{\RV}}
\newcommand{\LZpan}{\cL_{\ZZ_p\langle x \rangle}}
%

% Theories
\newcommand{\THen}{T_{\Hen}}
\newcommand{\Tpres}{T_{\Pres} }
\newcommand{\THeno}{T_{\Hen,0}}
%

% operators
\DeclareMathOperator{\acl}{acl}
\DeclareMathOperator{\dcl}{dcl}
\DeclareMathOperator{\rcl}{rcl}
\DeclareMathOperator{\qftp}{qftp}

\DeclareMathOperator{\ord}{ord}
\DeclareMathOperator{\ac}{ac}
\DeclareMathOperator{\rv}{rv}
\DeclareMathOperator{\res}{res}

\DeclareMathOperator{\dom}{dom}
\DeclareMathOperator{\lcm}{lcm}
\DeclareMathOperator{\id}{id}
\DeclareMathOperator{\ind}{ind}
\DeclareMathOperator{\lct}{lct}
\DeclareMathOperator{\graph}{graph}
\DeclareMathOperator{\sgn}{sgn}
\DeclareMathOperator{\rank}{rank}
\DeclareMathOperator{\fibdim}{fibdim}

\DeclareMathOperator{\cdim}{\#\mathrm{-}\dim}
\DeclareMathOperator{\radop}{radop}
\DeclareMathOperator{\radcl}{radcl}
\DeclareMathOperator{\LT}{LT}
\DeclareMathOperator{\Jac}{Jac}
\DeclareMathOperator{\tr}{tr}
%

%Big operators
\DeclareMathOperator{\Sym}{Sym}
\DeclareMathOperator{\Gal}{Gal}
\DeclareMathOperator{\Spec}{Spec}
\DeclareMathOperator{\Frac}{Frac}
\DeclareMathOperator{\Loc}{Loc}

\DeclareMathOperator{\Th}{Th}
\DeclareMathOperator{\Def}{Def}
\DeclareMathOperator{\Ob}{Ob}
\DeclareMathOperator{\K}{K}
\DeclareMathOperator{\Fn}{Fn}

\DeclareMathOperator{\Lone}{L^1}

%

%Binary operators
\newcommand{\divR}{\mathbin{|^{\cR}}}
%

%Brackets
\DeclarePairedDelimiter\floor{\lfloor}{\rfloor}
\DeclarePairedDelimiter{\ceil}{\lceil}{\rceil}
\DeclarePairedDelimiter{\abs}{\lvert}{\rvert}
\DeclarePairedDelimiter{\inprod}{<}{>}
\DeclarePairedDelimiter{\norm}{\lVert}{\rVert}

\DeclarePairedDelimiter{\bignorm}{\big\lVert}{\big\rVert}


\newtheorem{theorem}{Theorem}[section]
\newtheorem{lemma}[theorem]{Lemma}
\newtheorem{proposition}[theorem]{Proposition}
\newtheorem{corollary}[theorem]{Corollary}
\newtheorem{conjecture}[theorem]{Conjecture}
\newtheorem{lemma-definition}[theorem]{Lemma-Definition}

\theoremstyle{definition}
\newtheorem{definition}[theorem]{Definition}

\theoremstyle{remark}
\newtheorem{remark}[theorem]{Remark}
\newtheorem{example}[theorem]{Example}
\newtheorem{question}[theorem]{Question}
\newtheorem{notation}[theorem]{Notation}
\newtheorem{assumption}[theorem]{Assumption}


\begin{document}
Before we dive into the next theorem, recall that $\operatorname{Diag}(A)$ represents the diagram of $A$.
\begin{theorem}[3.1.4] \label{thm:3.1.4}
	Suppose that $L$ contains a constant symbol $c$, $T$ is an $L$-theory, and $\varphi(v)$ is an $L$-formula. The following are equivalent:
	\begin{enumerate}[label=\roman*)]
		\item There is a quantifier-free $L$-formula $\psi(v)$ such that $T\models \forall v\,(\varphi(v)\leftrightarrow \psi(v))$.
		\item If $M$ and $N$ are models of $T$, $A$ is an $L$-structure with $A\subseteq M$ and $A\subseteq N$, then $M\models \varphi(a)$ if and only if $N\models \varphi(a)$ for all $a\in A$.
	\end{enumerate}
\end{theorem}

\begin{proof}
	(i)$\Rightarrow$(ii): Suppose that $T\models \forall v\,(\varphi(v)\leftrightarrow \psi(v))$, where $\psi$ is quantifier-free. Let $a\in A$, where $A$ is a common substructure of $M$ and $N$ and the latter two structures are models of $T$. Quantifier-free formulas are preserved under substructure and extension, hence
	\[ M\models \varphi(a) \iff M\models \psi(a) \iff A\models \psi(a) \iff N\models \psi(a) \iff N\models \varphi(a). \]
	(ii)$\Rightarrow$(i): First, if $T\models \forall v\,\varphi(v)$, then $T\models \forall v\,(\varphi(v)\leftrightarrow c=c)$. Second, if $T\models \forall v\,\lnot\varphi(v)$, then $T\models \forall v\,(\varphi(v)\leftrightarrow c=c)$. Thus, we may assume that both $T\cup\{\varphi(v)\}$ and $T\cup\{\lnot\varphi(v)\}$ are satisfiable.
	
	Let $\Gamma(v)=\{\psi(v): \psi \text{ is quantifier-free and } T\models \forall v\,(\varphi(v)\to \psi(v))\}$. Let $d$ be (a tuple of) new constant symbol(s). We show that $T\cup \Gamma(d) \models \varphi(d)$. Then, by compactness, there are $\psi_1,\dots,\psi_n\in \Gamma$ such that
	\[ T\models \forall v\,\Big(\bigwedge_{i=1}^n \psi_i(v)\to \varphi(v)\Big), \]
	whence
	\[ T\models \forall v\,\Big(\bigwedge_{i=1}^n \psi_i(v) \leftrightarrow \varphi(v)\Big), \]
	and $\bigwedge_{i=1}^n \psi_i(v)$ is quantifier-free.
	
	\emph{Claim.} $T\cup \Gamma(d) \models \varphi(d)$. Suppose not. Let $M\models T\cup \Gamma(d)\cup\{\lnot\varphi(d)\}$. Let $A$ be the substructure of $M$ generated by $d$. Let $\Sigma = T\cup \operatorname{Diag}(A)\cup\{\varphi(d)\}$. If $\Sigma$ is unsatisfiable, then there are quantifier-free formulas $\psi_1(d),\dots,\psi_n(d)\in \operatorname{Diag}(A)$ such that
	\[ T\models \forall v\,\Big(\bigwedge_{i=1}^n \psi_i(v)\to \lnot\varphi(v)\Big). \]
	But then
	\[ T\models \forall v\,\Big(\varphi(v)\to \bigvee_{i=1}^n \lnot\psi_i(v)\Big), \]
	
	so $ \bigvee_{i=1}^n \lnot \psi_i(v) \in \Gamma$ and $A \models \bigvee_{i=1}^n \lnot \psi_i(d)$, a contradiction. Thus, $\Sigma$ is satisfiable.
	
	Let $N \models \Sigma$. Then $N \models \varphi(d)$. Because $\Sigma \supseteq \operatorname{Diag}(A)$, $A \subseteq N$, by Lemma [TODO]. But $M \models \lnot \phi(d)$; thus by {ii)}, $N \models \lnot \phi(d)$, a contradiction.
\end{proof}

%Rewritten to be closer to be closer to the natural induction on formulas in Mathlib
%Also, had to fix several errors introduced by chatgpt
\begin{lemma}[3.1.5] \label{lem:3.1.5}
	Let $T$ be an $L$-theory. Suppose that for every quantifier-free $L$-formula $\theta(v,w)$ there is a quantifier-free formula $\psi(v)$ such that
	\[ T\models \big(\exists w\,\theta(v,w) \leftrightarrow \psi(v)\big). \]
	Then $T$ has quantifier elimination.
\end{lemma}

\begin{proof}
	Let $\varphi(v)$ be an $L$-formula. We aim to show that $T\models \forall v\,(\varphi(v)\leftrightarrow \psi(v))$ for some quantifier-free $\psi(v)$. Proceed by induction on the complexity of $\varphi$.
	
	If $\varphi$ is quantifier-free, there is nothing to prove. Suppose inductively that for $i=0,1$ we have $T\models \forall v\,(\theta_i(v)\leftrightarrow \psi_i(v))$ with $\psi_i$ quantifier-free. If $\varphi(v)=\neg\theta_0(v)$, then
	\[ T\models \forall v\,(\varphi(v)\leftrightarrow \neg\psi_0(v)). \]
	
	If $\varphi(v)=\theta_0(v) \to  \theta_1(v)$, then
	\[ T\models \forall v\,(\varphi(v)\leftrightarrow (\psi_0(v)\to \psi_1(v))). \]
	In either case, $\phi$ is equivalent to a quantifier-free formula.
	
	Now suppose $\varphi(v)=\forall w\,\theta(v,w) \leftrightarrow \psi_0(v,w)$, where $\psi_0$ is quantifier-free and $\phi(v) = \forall w \theta(v,w)$ . 
	Then
	\[ T\models \forall v\,(\varphi(v)\leftrightarrow \forall w\,\psi_0(v,w)). \]
	Hence, 
	%
	\[ T \models \forall v\,(\lnot \varphi(v)\leftrightarrow \exists \lnot w\,\psi_0(v,w)). \]
	By hypothesis, there is quantifier-free $\psi(v)$ with
	\[ T\models \forall v\,(\lnot \forall w\,\theta(v,w)\leftrightarrow \psi(v)). \]
	It follows that 
	%
	\[ T\models \forall v\,( \forall w\,\theta(v,w)\leftrightarrow \lnot \psi(v)), \]
	%
	where $\lnot \psi(v)$ is quantifier-free.
\end{proof}

\begin{corollary}[3.1.6]
	Let $T$ be an $L$-theory. Suppose that for all quantifier-free formulas $\varphi(v,w)$, if $M,N\models T$, $A$ is a common substructure of $M$ and $N$, $a\in A$, and there is $b\in M$ such that $M\models \varphi(a,b)$, then there is $c\in N$ such that $N\models \varphi(a,c)$. Then $T$ has quantifier elimination.
\end{corollary}
%
\begin{proof}
	Combine Theorem~\ref{thm:3.1.4} and Lemma~\ref{lem:3.1.5}.
\end{proof}

\end{document}