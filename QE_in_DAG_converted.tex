
\documentclass[11pt]{article}
\usepackage[utf8]{inputenc}
\usepackage{amsmath,amssymb,amsthm}
\usepackage{mathtools}
\usepackage{microtype}
\usepackage{enumitem}
\usepackage{geometry}
\geometry{margin=1in}
\title{Quantifier Elimination in Dense Algebraic/Ordered Structures}
\author{(Converted from PDF)}
\date{}


\newcommand{\BB}{\mathbb{B}}
\newcommand{\CC}{\mathbb{C}}
\newcommand{\DD}{\mathbb{D}}
\newcommand{\EE}{\mathbb{E}}
\newcommand{\FF}{\mathbb{F}}
\newcommand{\GG}{\mathbb{G}}
\newcommand{\HH}{\mathbb{H}}
\newcommand{\II}{\mathbb{I}}
\newcommand{\JJ}{\mathbb{J}}
\newcommand{\KK}{\mathbb{K}}
\newcommand{\LL}{\mathbb{L}}
\newcommand{\MM}{\mathbb{M}}
\newcommand{\NN}{\mathbb{N}}
\newcommand{\OO}{\mathbb{O}}
\newcommand{\PP}{\mathbb{P}}
\newcommand{\QQ}{\mathbb{Q}}
\newcommand{\RR}{\mathbb{R}}
%\newcommand{\SS}{\mathbb{S}}
\newcommand{\TT}{\mathbb{T}}
\newcommand{\UU}{\mathbb{U}}
\newcommand{\VV}{\mathbb{V}}
\newcommand{\WW}{\mathbb{W}}
\newcommand{\XX}{\mathbb{X}}
\newcommand{\YY}{\mathbb{Y}}
\newcommand{\ZZ}{\mathbb{Z}}


\newcommand{\cA}{\mathcal{A}}
\newcommand{\cB}{\mathcal{B}}
\newcommand{\cC}{\mathcal{C}}
\newcommand{\cD}{\mathcal{D}}
\newcommand{\cE}{\mathcal{E}}
\newcommand{\cF}{\mathcal{F}}
\newcommand{\cG}{\mathcal{G}}
\newcommand{\cH}{\mathcal{H}}
\newcommand{\cI}{\mathcal{I}}
\newcommand{\cJ}{\mathcal{J}}
\newcommand{\cK}{\mathcal{K}}
\newcommand{\cL}{\mathcal{L}}
\newcommand{\cM}{\mathcal{M}}
\newcommand{\cN}{\mathcal{N}}
\newcommand{\cO}{\mathcal{O}}
\newcommand{\cP}{\mathcal{P}}
\newcommand{\cQ}{\mathcal{Q}}
\newcommand{\cR}{\mathcal{R}}
\newcommand{\cS}{\mathcal{S}}
\newcommand{\cT}{\mathcal{T}}
\newcommand{\cU}{\mathcal{U}}
\newcommand{\cV}{\mathcal{V}}
\newcommand{\cW}{\mathcal{W}}
\newcommand{\cX}{\mathcal{X}}
\newcommand{\cY}{\mathcal{Y}}
\newcommand{\cZ}{\mathcal{Z}}
%
\newcommand{\bA}{\mathbb{A}}
\newcommand{\bB}{\mathbb{B}}
\newcommand{\bC}{\mathbb{C}}
\newcommand{\bD}{\mathbb{D}}
\newcommand{\bE}{\mathbb{E}}
\newcommand{\bF}{\mathbb{F}}
\newcommand{\bG}{\mathbb{G}}
\newcommand{\bH}{\mathbb{H}}
\newcommand{\bI}{\mathbb{I}}
\newcommand{\bJ}{\mathbb{J}}
\newcommand{\bK}{\mathbb{K}}
\newcommand{\bL}{\mathbb{L}}
\newcommand{\bM}{\mathbb{M}}
\newcommand{\bN}{\mathbb{N}}
\newcommand{\bO}{\mathbb{O}}
\newcommand{\bP}{\mathbb{P}}
\newcommand{\bQ}{\mathbb{Q}}
\newcommand{\bR}{\mathbb{R}}
\newcommand{\bS}{\mathbb{S}}
\newcommand{\bT}{\mathbb{T}}
\newcommand{\bU}{\mathbb{U}}
\newcommand{\bV}{\mathbb{V}}
\newcommand{\bW}{\mathbb{W}}
\newcommand{\bX}{\mathbb{X}}
\newcommand{\bY}{\mathbb{Y}}
\newcommand{\bZ}{\mathbb{Z}}



% sub- and superscripts
\newcommand{\an}{ \mathrm{an} }
\newcommand{\val}{ \mathrm{val} }
\newcommand{\ring}{ \mathrm{ring} }
\newcommand{\og}{ \mathrm{og} }
\newcommand{\Pres}{ \mathrm{Pres} }
\newcommand{\gDP}{\mathrm{gDP}}
\newcommand{\DP}{ \mathrm{DP}}
\newcommand{\Hen}{\mathrm{Hen}}
\newcommand{\alg}{\mathrm{alg}}
\newcommand{\trans}{\mathrm{trans}}
\newcommand{\fine}{\mathrm{fine}}
\newcommand{\oag}{\mathrm{oag}}
\newcommand{\ovf}{\mathrm{ovf}}
\newcommand{\trop}{\mathrm{trop}}
\newcommand{\RRV}{\mathrm{RRV}}
\newcommand{\omin}{\mathrm{omin}}
\newcommand{\eq}{\mathrm{eq}}
%

% Sorts
\newcommand{\RV}{\mathrm{RV}}
\newcommand{\RF}{\mathrm{RF}}
\newcommand{\RVprod}{ \RV_{\bar{n}} }
\newcommand{\VG}{\mathrm{VG}}
\newcommand{\VF}{\mathrm{VF}}
%

% Languages
\newcommand{\lang}{\mathcal{L}}
\newcommand{\Lval}{\lang_{\val}}
\newcommand{\Lor}{\lang_{\mathrm{or}}}
\newcommand{\Lan}{\lang_{\an}}
\newcommand{\Lring}{\lang_{\ring}}
\newcommand{\Lvf}{\lang_{\VF}}
\newcommand{\LRV}{ \lang^{\RV} }
\newcommand{\LRVplus}{\lang^{\RV^{+}}}
\newcommand{\Lac}{\lang^{\ac}}
\newcommand{\Log}{\lang_{\og}}
\newcommand{\Lpres}{\lang_{\Pres}}
\newcommand{\LDP}{\lang_{\DP}}
\newcommand{\LgDP}{\lang_{\gDP}}
\newcommand{\Lgroup}{\lang_{\mathrm{group}} }
\newcommand{\Ldpr}{\lang_{\gDP}^{\mathrm{res}}}
\newcommand{\Lrfn}{\lang_{\RF_N}}
\newcommand{\Lvg}{\lang_{\VG}}
\newcommand{\Lrv}{\lang{\RV}}
\newcommand{\LZpan}{\cL_{\ZZ_p\langle x \rangle}}
%

% Theories
\newcommand{\THen}{T_{\Hen}}
\newcommand{\Tpres}{T_{\Pres} }
\newcommand{\THeno}{T_{\Hen,0}}
%

% operators
\DeclareMathOperator{\acl}{acl}
\DeclareMathOperator{\dcl}{dcl}
\DeclareMathOperator{\rcl}{rcl}
\DeclareMathOperator{\qftp}{qftp}

\DeclareMathOperator{\ord}{ord}
\DeclareMathOperator{\ac}{ac}
\DeclareMathOperator{\rv}{rv}
\DeclareMathOperator{\res}{res}

\DeclareMathOperator{\dom}{dom}
\DeclareMathOperator{\lcm}{lcm}
\DeclareMathOperator{\id}{id}
\DeclareMathOperator{\ind}{ind}
\DeclareMathOperator{\lct}{lct}
\DeclareMathOperator{\graph}{graph}
\DeclareMathOperator{\sgn}{sgn}
\DeclareMathOperator{\rank}{rank}
\DeclareMathOperator{\fibdim}{fibdim}

\DeclareMathOperator{\cdim}{\#\mathrm{-}\dim}
\DeclareMathOperator{\radop}{radop}
\DeclareMathOperator{\radcl}{radcl}
\DeclareMathOperator{\LT}{LT}
\DeclareMathOperator{\Jac}{Jac}
\DeclareMathOperator{\tr}{tr}
%

%Big operators
\DeclareMathOperator{\Sym}{Sym}
\DeclareMathOperator{\Gal}{Gal}
\DeclareMathOperator{\Spec}{Spec}
\DeclareMathOperator{\Frac}{Frac}
\DeclareMathOperator{\Loc}{Loc}

\DeclareMathOperator{\Th}{Th}
\DeclareMathOperator{\Def}{Def}
\DeclareMathOperator{\Ob}{Ob}
\DeclareMathOperator{\K}{K}
\DeclareMathOperator{\Fn}{Fn}

\DeclareMathOperator{\Lone}{L^1}

%

%Binary operators
\newcommand{\divR}{\mathbin{|^{\cR}}}
%

%Brackets
\DeclarePairedDelimiter\floor{\lfloor}{\rfloor}
\DeclarePairedDelimiter{\ceil}{\lceil}{\rceil}
\DeclarePairedDelimiter{\abs}{\lvert}{\rvert}
\DeclarePairedDelimiter{\inprod}{<}{>}
\DeclarePairedDelimiter{\norm}{\lVert}{\rVert}

\DeclarePairedDelimiter{\bignorm}{\big\lVert}{\big\rVert}


\newtheorem{theorem}{Theorem}[section]
\newtheorem{lemma}[theorem]{Lemma}
\newtheorem{proposition}[theorem]{Proposition}
\newtheorem{corollary}[theorem]{Corollary}
\newtheorem{conjecture}[theorem]{Conjecture}
\newtheorem{lemma-definition}[theorem]{Lemma-Definition}

\theoremstyle{definition}
\newtheorem{definition}[theorem]{Definition}

\theoremstyle{remark}
\newtheorem{remark}[theorem]{Remark}
\newtheorem{example}[theorem]{Example}
\newtheorem{question}[theorem]{Question}
\newtheorem{notation}[theorem]{Notation}
\newtheorem{assumption}[theorem]{Assumption}


\begin{document}
\maketitle

\section{Algebraic Examples}

This .tex file is about the model theory of divisilbe abelian groups. We define the notion of quantifier elimination (QE) and show that the first order theory of divisible abelian groups (DAG) has quantifier elimination.

\subsection{Quantifier Elimination}

We have the following helper proposition.

Let $M,N$ be $L$-structures for a first-order Language $L$.
\begin{proposition}[1.1.8] \label{prop:1.1.8}
	Suppose that $M$ is a substructure of $N$, $a\in M$, and $\varphi(v)$ is a quantifier\mbox{-}free formula. Then
	\[
	M \models \varphi(a) \quad\text{if and only if}\quad N \models \varphi(a).
	\]
\end{proposition}
\begin{proof}
	
	\emph{Claim.} If $t(v)$ is a term and $b\in M$, then $t^M(b)=t^N(b)$. This is proved by induction on terms. 
	If $t$ is the constant symbol $c$, then $c^M=c^N$. If $t$ is the variable $v_i$, then $t^M(b)=b_i=t^N(b)$. 
	Suppose that $t=f(t_1,\dots,t_n)$, where $f$ is an $n$-ary function symbol and $t_1,\dots,t_n$ are terms, and $t_i^M(b)=t_i^N(b)$ for $i=1,\dots,n$. Because $M\subseteq N$, we have $f^M = f^N\!\upharpoonright M^n$. Thus,
	\[
	t^M(b)=f^M\big(t_1^M(b),\dots,t_n^M(b)\big)
	= f^N\big(t_1^M(b),\dots,t_n^M(b)\big)
	= f^N\big(t_1^N(b),\dots,t_n^N(b)\big)
	= t^N(b).
	\]
	This proves the claim.
	
	We now prove the proposition by induction on formulas. If $\varphi$ is $t_1=t_2$, then
	\[
	M \models \varphi(a) \iff t_1^M(a)=t_2^M(a) \iff t_1^N(a)=t_2^N(a) \iff N \models \varphi(a).
	\]
	If $\varphi$ is $R(t_1,\dots,t_n)$, where $R$ is an $n$-ary relation symbol, then
	\[
	\begin{aligned}
		M \models \varphi(a) 
		&\iff \big(t_1^M(a),\dots,t_n^M(a)\big)\in R^M \\
		&\iff \big(t_1^M(a),\dots,t_n^M(a)\big)\in R^N \\
		&\iff \big(t_1^N(a),\dots,t_n^N(a)\big)\in R^N \\
		&\iff N \models \varphi(a).
	\end{aligned}
	\]
	Thus, the proposition is true for all atomic formulas.
	
	Suppose that the proposition is true for $\psi$ and that $\varphi$ is $\neg\psi$. Then
	\[
	M \models \neg\varphi(a) \iff M \models \psi(a) \iff N \models \psi(a) \iff N \models \varphi(a).
	\]
	Finally, suppose that the proposition is true for $\psi_0$ and $\psi_1$, and that $\varphi$ is $\psi_0 \wedge \psi_1$. Then
	\[
	M \models \varphi(a) \iff M \models \psi_0(a)\ \text{and}\ M \models \psi_1(a)
	\iff N \models \psi_0(a)\ \text{and}\ N \models \psi_1(a)
	\iff N \models \varphi(a).
	\]
	
	We have shown that the proposition holds for all atomic formulas and that if it holds for $\varphi$ and $\psi$, then it also holds for $\neg\varphi$ and $\varphi \wedge \psi$. Because the set of quantifier\mbox{-}free formulas is the smallest set of formulas containing the atomic formulas and closed under negation and conjunction, the proposition is true for all quantifier\mbox{-}free formulas. \qedhere
\end{proof}


Let $L$ be a first-order language and $T$ and $L$-theory.

\begin{definition}[Quantifier Elimination]
We say that a theory $T$ has \emph{quantifier elimination} if for every formula $\varphi$ there is a quantifier-free formula $\psi$ such that
$T \models (\varphi \leftrightarrow \psi)$.
\end{definition}


Before we dive into the next theorem, recall that $\operatorname{Diag}(A)$ represents the diagram of $A$.
\begin{theorem}[3.1.4] \label{thm:3.1.4}
	Suppose that $L$ contains a constant symbol $c$, $T$ is an $L$-theory, and $\varphi(v)$ is an $L$-formula. The following are equivalent:
	\begin{enumerate}[label=\roman*)]
		\item There is a quantifier-free $L$-formula $\psi(v)$ such that $T\models \forall v\,(\varphi(v)\leftrightarrow \psi(v))$.
		\item If $M$ and $N$ are models of $T$, $A$ is an $L$-structure with $A\subseteq M$ and $A\subseteq N$, then $M\models \varphi(a)$ if and only if $N\models \varphi(a)$ for all $a\in A$.
	\end{enumerate}
\end{theorem}

\begin{proof}
	(i)$\Rightarrow$(ii): Suppose that $T\models \forall v\,(\varphi(v)\leftrightarrow \psi(v))$, where $\psi$ is quantifier-free. Let $a\in A$, where $A$ is a common substructure of $M$ and $N$ and the latter two structures are models of $T$. Quantifier-free formulas are preserved under substructure and extension, hence
	\[ M\models \varphi(a) \iff M\models \psi(a) \iff A\models \psi(a) \iff N\models \psi(a) \iff N\models \varphi(a). \]
	(ii)$\Rightarrow$(i): First, if $T\models \forall v\,\varphi(v)$, then $T\models \forall v\,(\varphi(v)\leftrightarrow c=c)$. Second, if $T\models \forall v\,\lnot\varphi(v)$, then $T\models \forall v\,(\varphi(v)\leftrightarrow c=c)$. Thus, we may assume that both $T\cup\{\varphi(v)\}$ and $T\cup\{\lnot\varphi(v)\}$ are satisfiable.
	
	Let $\Gamma(v)=\{\psi(v): \psi \text{ is quantifier-free and } T\models \forall v\,(\varphi(v)\to \psi(v))\}$. Let $d$ be (a tuple of) new constant symbol(s). We show that $T\cup \Gamma(d) \models \varphi(d)$. Then, by compactness, there are $\psi_1,\dots,\psi_n\in \Gamma$ such that
	\[ T\models \forall v\,\Big(\bigwedge_{i=1}^n \psi_i(v)\to \varphi(v)\Big), \]
	whence
	\[ T\models \forall v\,\Big(\bigwedge_{i=1}^n \psi_i(v) \leftrightarrow \varphi(v)\Big), \]
	and $\bigwedge_{i=1}^n \psi_i(v)$ is quantifier-free.
	
	\emph{Claim.} $T\cup \Gamma(d) \models \varphi(d)$. Suppose not. Let $M\models T\cup \Gamma(d)\cup\{\lnot\varphi(d)\}$. Let $A$ be the substructure of $M$ generated by $d$. Let $\Sigma = T\cup \operatorname{Diag}(A)\cup\{\varphi(d)\}$. If $\Sigma$ is unsatisfiable, then there are quantifier-free formulas $\psi_1(d),\dots,\psi_n(d)\in \operatorname{Diag}(A)$ such that
	\[ T\models \forall v\,\Big(\bigwedge_{i=1}^n \psi_i(v)\to \lnot\varphi(v)\Big). \]
	But then
	\[ T\models \forall v\,\Big(\varphi(v)\to \bigvee_{i=1}^n \lnot\psi_i(v)\Big), \]
	
	so $ \bigvee_{i=1}^n \lnot \psi_i(v) \in \Gamma$ and $A \models \bigvee_{i=1}^n \lnot \psi_i(d)$, a contradiction. Thus, $\Sigma$ is satisfiable.
	
	Let $N \models \Sigma$. Then $N \models \varphi(d)$. Because $\Sigma \supseteq \operatorname{Diag}(A)$, $A \subseteq N$, by Lemma [TODO]. But $M \models \lnot \phi(d)$; thus by {ii)}, $N \models \lnot \phi(d)$, a contradiction.
\end{proof}

%Rewritten to be closer to be closer to the natural induction on formulas in Mathlib
%Also, had to fix several errors introduced by chatgpt
\begin{lemma}[3.1.5] \label{lem:3.1.5}
	Let $T$ be an $L$-theory. Suppose that for every quantifier-free $L$-formula $\theta(v,w)$ there is a quantifier-free formula $\psi(v)$ such that
	\[ T\models \big(\exists w\,\theta(v,w) \leftrightarrow \psi(v)\big). \]
	Then $T$ has quantifier elimination.
\end{lemma}

\begin{proof}
	Let $\varphi(v)$ be an $L$-formula. We aim to show that $T\models \forall v\,(\varphi(v)\leftrightarrow \psi(v))$ for some quantifier-free $\psi(v)$. Proceed by induction on the complexity of $\varphi$.
	
	If $\varphi$ is quantifier-free, there is nothing to prove. Suppose inductively that for $i=0,1$ we have $T\models \forall v\,(\theta_i(v)\leftrightarrow \psi_i(v))$ with $\psi_i$ quantifier-free. If $\varphi(v)=\neg\theta_0(v)$, then
	\[ T\models \forall v\,(\varphi(v)\leftrightarrow \neg\psi_0(v)). \]
	
	If $\varphi(v)=\theta_0(v) \to  \theta_1(v)$, then
	\[ T\models \forall v\,(\varphi(v)\leftrightarrow (\psi_0(v)\to \psi_1(v))). \]
	In either case, $\phi$ is equivalent to a quantifier-free formula.
	
	Now suppose $\varphi(v)=\forall w\,\theta(v,w) \leftrightarrow \psi_0(v,w)$, where $\psi_0$ is quantifier-free and $\phi(v) = \forall w \theta(v,w)$ . 
	Then
	\[ T\models \forall v\,(\varphi(v)\leftrightarrow \forall w\,\psi_0(v,w)). \]
	Hence, 
	%
	\[ T \models \forall v\,(\lnot \varphi(v)\leftrightarrow \exists \lnot w\,\psi_0(v,w)). \]
	By hypothesis, there is quantifier-free $\psi(v)$ with
	\[ T\models \forall v\,(\lnot \forall w\,\theta(v,w)\leftrightarrow \psi(v)). \]
	It follows that 
	%
	\[ T\models \forall v\,( \forall w\,\theta(v,w)\leftrightarrow \lnot \psi(v)), \]
	%
	where $\lnot \psi(v)$ is quantifier-free.
\end{proof}

\begin{corollary}[3.1.6] \label{cor:3.1.6}
	Let $T$ be an $L$-theory. Suppose that for all quantifier-free formulas $\varphi(v,w)$, if $M,N\models T$, $A$ is a common substructure of $M$ and $N$, $a\in A$, and there is $b\in M$ such that $M\models \varphi(a,b)$, then there is $c\in N$ such that $N\models \varphi(a,c)$. Then $T$ has quantifier elimination.
\end{corollary}
%
\begin{proof}
	Combine Theorem~\ref{thm:3.1.4} and Lemma~\ref{lem:3.1.5}.
\end{proof}

\subsection*{Divisible Abelian Groups}
It is convenient to work in the language $L=\{+, -, 0\}$, since in this language substructures of groups are groups. Let $\mathrm{DAG}$ be the $L$-theory of nontrivial torsion-free divisible Abelian groups. We show that $\mathrm{DAG}$ has quantifier elimination.

\begin{definition}
	The language $\{+,-,0\}$ of Abelian groups consists of 
	\begin{enumerate}
		\item a binary function symbol $+$,
		\item a unary function symbol  $-$,
		\item a constant symbol $0$.
	\end{enumerate}
\end{definition}

\begin{definition}
	Any Abelian group $G$ is naturally an $\{+,-,0\}$-structure, where $+$ is interpreted as the group operation, $-$ as inversion and $0$ as the unit element of the group.
\end{definition}

\begin{lemma}[3.1.7]
	Suppose $G$ and $H$ are nontrivial torsion-free divisible Abelian groups with $G\subseteq H$, $\psi(v,w)$ is quantifier-free, $a\in G$, $b\in H$, and $H\models \psi(a,b)$. Then there is $c\in G$ such that $G\models \psi(a,c)$.
\end{lemma}

\begin{proof}
	Write $\psi$ in disjunctive normal form (a finite disjunction of finite conjunctions of atomic or negated atomic formulas). Since $H\models \psi(a,b)$, there is a conjunctive clause true at $(a,b)$. Thus we may assume $\psi$ is a conjunction of atomic and negated atomic formulas.
	
	In the language of Abelian groups, an atomic formula in variables $(v_1,\dots,v_m,w)$ is of the form
	\[ \sum_{j=1}^m n_j v_j + m\,w = 0, \]
	for some integers $n_1,\dots,n_m,m$. Hence we may write our clause as
	\[ \bigwedge_{i\in I} \big( g_i + m_i w = 0 \big)\ \wedge\ \bigwedge_{j\in J} \big( h_j + n_j w \neq 0 \big), \]
	where $g_i,h_j\in G$ and $m_i,n_j\in \ZZ$.
	
	If $I\neq \emptyset$, the equalities force $w$ to be a specific element. For each $i\in I$ we have $w=-\tfrac{1}{m_i}g_i$. Consistency in $H$ implies all these values agree; call the common value $b_0\in H$. Since $G$ is divisible and $g_i\in G$, we have $-\tfrac{1}{m_i}g_i\in G$, hence $b_0\in G$. Moreover, $H\models$ the disequalities imply $b_0\ne -\tfrac{1}{n_j}h_j$ for all $j\in J$, and these values lie in $G$ as well; thus $G\models$ the same clause with $w=b_0$.
	
	If $I=\emptyset$, the clause becomes a finite conjunction of \emph{disequalities} $w\ne -\tfrac{1}{n_j}h_j$. The set of forbidden values is finite; since $G$ is infinite (torsion-free divisible groups are infinite), we can choose $c\in G$ avoiding all forbidden values, and then $G\models$ the clause with $w=c$.
	
	Therefore there exists $c\in G$ such that $G\models \psi(a,c)$.
\end{proof}

\begin{lemma}[3.1.8; Divisible Hull] \label{lem:divisible-hull}
	Suppose $G$ is a torsion-free Abelian group. Then there exists a torsion-free divisible Abelian group $H$ (called the \emph{divisible hull} of $G$) and an embedding $i\colon G\to H$ such that if $j\colon G\to H'$ is an embedding of $G$ into a torsion-free divisible Abelian group $H'$, then there is an embedding $h\colon H\to H'$ with $j=h\circ i$.
\end{lemma}

\begin{proof}
	This is a well-known theorem in group theory. We repeat the proof here for convenience.
	If $G$ is trivial, take $H=\QQ$. Otherwise, define
	\[ X=\{(g,n): g\in G,\ n\in \NN,\ n>0\}, \]
	with the intuition that $(g,n)$ represents $g/n$. Define an equivalence relation $\sim$ on $X$ by
	\[ (g,n)\sim (h,m) \iff m g = n h. \]
	Let $H=X/\sim$, and denote an equivalence class by $[(g,n)]$. Define the group operation by
	\[ [(g,n)]+[(h,m)]=[(m g + n h,\, m n)]. \]
	One verifies this is well-defined: if $(g_0,n_0)\sim (g,n)$, then
	\[ (m g_0 + n_0 h,\, m n_0) \sim (m g + n h,\, m n). \]
	Similarly, define
	\[ -[(g,n)]=[(-g,n)], \qquad [(g,n)]-[(h,m)]=[(m g - n h,\, m n)]. \]
	Then $(H,+)$ is an Abelian group with identity $[(0,1)]$ and inverse $-[(g,n)]$.
	
	Torsion-freeness: if $n>0$ and $n[(g,m)]=[(0,k)]$ for some $k>0$, then $k n g=0$ in $G$. Since $G$ is torsion-free, $g=0$, and hence $[(g,m)]=[(0,1)]$.
	
	Divisibility: for $n>0$,
	\[ n[(g,m n)]=[(n g, m n)]=[(g,m)], \]
	so every element has an $n$-th root.
	
	Embed $G$ into $H$ via $i(g)=[(g,1)]$. This is a group embedding since
	\[ [(g,1)]+[(h,1)]=[(g+h,1)]. \]
	Finally, given an embedding $j\colon G\to H'$ into a torsion-free divisible Abelian group $H'$, define $h\colon H\to H'$ by
	\[ h\big([(g,n)]\big)=\frac{1}{n}\,j(g). \]
	This is well-defined and a group embedding, and clearly $j=h\circ i$.
\end{proof}

\begin{theorem}[3.1.9]
	$\mathrm{DAG}$ has quantifier elimination.
\end{theorem}

\begin{proof}
	Let $G_0$ and $G_1$ be torsion-free divisible Abelian groups, and let $G$ be a common subgroup of $G_0$ and $G_1$. Fix $g\in G$, and let $\varphi(v,w)$ be quantifier-free. Suppose $G_0\models \varphi(g,h)$ for some $h\in G_0$. Let $H$ be the divisible hull of $G$. Since we can embed $H$ into $G_0$, Lemma~\ref{lem:divisible-hull} yields $H\models \exists w\,\varphi(g,w)$. Embedding $H$ into $G_1$ gives some $h'\in G_1$ with $G_1\models \varphi(g,h')$. By Corollary~\ref{cor:3.1.6}, $\mathrm{DAG}$ has quantifier elimination.
\end{proof}

\end{document}
